\documentclass{article}
\title{The Tragedy of Elections}
\date{2016-6-2}
\author{Vincent Yang}
\begin{document}
  \pagenumbering{gobble} %no numbers
  \maketitle
  \newpage
  \pagenumbering{arabic}
  \section{Abstract}
  Delta Sigma Pi - Nu Rho Chapter suffers from (1) two structural flaws that affect the (potential) members,
  and (2) two ritual flaws that prevent effective/efficient meetings - the exact opposite of their original purpose.
  \paragraph{Problems for potential members}First, there is insufficient %(1)
  knowledge for people to know whether or not to add information. Second, People are not responsible
  for their answers. Overall, this results in an inconsistent standard for pledges.\\
  My proposal is to have people vote publically. This forces people to 
  \begin{enumerate}
    \item Have substantial evidence/reasoning for voting one way or another
    \item Speak up to defend their opinion, when they see that a voting may not be going as desired, ultimately educating everyone and giving more information with 
      which people can make better decisions
    \item Stand up for themselves. If you are the type to succumb to peer pressure in your vote, then you should not be voting against the majority in the first place.
  \end{enumerate}

  \paragraph{Problems that prevent effective and efficient meetings}
  Meetings should be run in an effective and efficient manner. Counting useless votes and
  enforcing strict 3-minute policies goes against the purpose of the meeting.\\
  My solution is to have soft deadlines and to only count the votes that matter, publically.

  \section{The Problem in Choosing Potential Members}
    \paragraph{The Implied Qualifications} %You have to be this good, no quota
    \paragraph{The Double Standard} %self explanetary
      The flaw is this and this and this
      and this and this
    \paragraph{Gamification}
      Gamification (n): The application of game theory concepts and techniques to non-game activities.
      Game Theory (n): A branch of mathematics that seeks to understand why an individual makes a particular
      decision and how the decisions made by one individual affect others.
      Game (n): A procedure or strategy for gaining an end.
      \\Source: http://searchsalesforce.techtarget.com/definition/gamification
  \section{Ritual}
  \section{The Solution for Choosing Potential Members}
    \paragraph{Public Voting}
    \paragraph{Effects}

  \section{The Problem in Counting Useless Votes} %(2)
    \paragraph{The Importance of Time}
  \section{Ritual}
  \section{The Solution to Inefficient Procedure}
    \paragraph{}
    \paragraph{Effects With Regards to Public Voting}
  \section{Conclusion}
\end{document}
